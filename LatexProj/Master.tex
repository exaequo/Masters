\documentclass[a4paper,12pt,numbers=noenddot]{report}

\usepackage{polski}
\usepackage[utf8]{inputenc}
\usepackage{graphicx}
\usepackage{cite}
\usepackage{float}
\usepackage{url} 
\usepackage{hyperref}
\usepackage{amsmath}
\usepackage{gensymb}
\usepackage{pdfpages}
\usepackage[nottoc,numbib]{tocbibind}
\usepackage{etoolbox}
\usepackage{enumitem}

\usepackage{titlesec, blindtext, color}

\titleformat{\chapter}[hang]{\Huge\bfseries}{\thechapter{. }}{0pt}{\Huge\bfseries}


\usepackage[
  inner = 3cm,
  outer = 2.5cm,
  top = 2.5cm,
  bottom = 2.5cm,
]{geometry}


\title{TYTUŁ pracy}
\date{\today}
\author{Hubert Marcinkowski}

\begin{document}
	\nocite{*}
	\pagenumbering{gobble}
	\maketitle
	%\includepdf{title2.pdf}

	\newpage
	\pagenumbering{arabic}
	\tableofcontents
	\newpage
	%\setcitestyle{square}	
	
\chapter{Wstęp}
Projektowanie interfejsów jest tematyką bardzo złożoną i nadzwyczaj kluczową w procesie tworzenia aplikacji, w tym, między innymi, gier komputerowych. Złe zaprojektowanie tej części w prosty sposób może przyczynić się do całkowitego braku zrozumienia całego przesłania gry przez użytkowników. Istnieje wiele metodologii mających na celu analizę jakości stworzonego projektu - od metod całkowicie subiektywnych po takie, które stawiają za zadanie sprowadzenie jak największej części elementów do postaci parametrycznej. 

- Trudne zadanie związane z tym, że poszukiwany jest pewien zbiór reguł, który ułatwiłby proces projektowania. 
- Część z przedstawianych rozwiązań wybitnie nie pasuje do dziedziny mobilek
- Problemy związane z projektowaniem dla mobilek
- Przygotowane zostały heurystyki, które mają na celu dokładne określenie jakie warunki aplikacja musi spełniać, by mogła zostać dobrze odebrana przez użytkowników
- coś o tym, że użytkownik w czasie gry często zostaje postawiony przed przyswojeniem bardzo abstrakcyjnego zadania w celu zrozumienia, a co za tym idzie grania w daną grę

W pracy tej opisane zostało badanie mające na celu zweryfikowanie wpływ spełniania tych heurystyk na pozna

\chapter{Cel, założenia oraz zakres pracy}
Celem tej pracy było zbadanie wpływu określonych heurystyk tworzenia interfejsów na jakość przyswajania nowych mechanik gry mobilnej. Zbadane zostały wybrane dwie opisane przez H. Desuvire oraz Ch. Wiberg [odnośnik!] reguły, które odnoszą się do użyteczności projektowanych aplikacji.

Przeprowadzone zostało doświadczenie, które miało na celu określenie, czy gra, której interfejs spełnia zadane warunki jest lepiej rozumiana i szybciej opanowywana od takiej, której nie są spełnione wybrane heurystyki. Przygotowana została w tym celu aplikacja na urządzenia mobilne, w której użytkownikom postawione zostają zadania, które nie wpisują się w popularne szablony obecnych gier na te platformy. Wyróżnione zostały cztery różne wersje interfejsów przedstawianych użytkownikowi, każda wersja spełniała inny złożenie heurystyk.
\chapter{Interfejsy gier mobilnych}
Tutaj opisuję dokładnie problemy związane
\chapter{Analiza podobnych rozwiązań}
Tutaj opisuję inne rozwiązania? badania?
\chapter{Metoda}
	\section{Sphaze}
Opis sphaze
		\subsection{Mechaniki gry}
		\subsection{Samouczki}
	\section{Hipotezy}
Hipoteza 1.

Hipoteza 2.

	\section{Badanie}
Pierwsza hipoteza
Druga hipoteza

\chapter{Wyniki badań}
Tu przedstawiam wyniki badań
\chapter{Wnioski}
Tu zapiszę wnioski
\chapter{Zakończenie}
Tu podsumuję?

%\pagenumbering{gobble}

\chapter{Zawartość płyty}
\begin{enumerate}[label={[\arabic*]}]
  \item Tekst pracy w formacie PDF
  \item Pliki z wynikami przeprowadzonych badań
  \item Plik z wynikami przeprowadzonej analizy
\end{enumerate}

%TODO: Bibliografia
\end{document}
