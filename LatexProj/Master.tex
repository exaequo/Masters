\documentclass[a4paper,12pt,numbers=noenddot]{report}

\usepackage{polski}
\usepackage[utf8]{inputenc}
\usepackage{graphicx}
\usepackage{cite}
\usepackage{float}
\usepackage{url} 
\usepackage{hyperref}
\usepackage{amsmath}
\usepackage{gensymb}
\usepackage{pdfpages}
\usepackage[nottoc,numbib]{tocbibind}
\usepackage{etoolbox}
\usepackage{enumitem}

\usepackage{titlesec, blindtext, color}

\titleformat{\chapter}[hang]{\Huge\bfseries}{\thechapter{. }}{0pt}{\Huge\bfseries}


\usepackage[
  inner = 3cm,
  outer = 2.5cm,
  top = 2.5cm,
  bottom = 2.5cm,
]{geometry}


\title{TYTUŁ pracy}
\date{\today}
\author{Hubert Marcinkowski}

\begin{document}
	\nocite{*}
	\pagenumbering{gobble}
	\maketitle
	%\includepdf{title2.pdf}

	\newpage
	\pagenumbering{arabic}
	\tableofcontents
	\newpage
	%\setcitestyle{square}	
	
\chapter{Wstęp}
Magisterka, czas zacząć
\chapter{Wyjaśnienie pojęć}
Wyjaśniam
\chapter{Metoda}
Tu daję metodę, hipotezę
	\section{Hipotezy}
Pierwsza hipoteza
Druga hipoteza

\chapter{Wyniki badań}
Tu przedstawiam wyniki badań
\chapter{Wnioski}
Tu zapiszę wnioski
\chapter{Zakończenie}
Tu podsumuję?

%\pagenumbering{gobble}

\chapter{Zawartość płyty}
\begin{enumerate}[label={[\arabic*]}]
  \item Tekst pracy w formacie PDF
  \item Pliki z wynikami przeprowadzonych badań
  \item Plik z wynikami przeprowadzonej analizy
\end{enumerate}

%TODO: Bibliografia
\end{document}
